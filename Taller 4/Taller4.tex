\documentclass[10pt, twocolumn]{article}
%\documentclass[prd,amssymb,preprint]{revtex4}
\setlength{\topmargin}{-0.7in}
\setlength{\oddsidemargin}{-0.4in} 
\setlength{\textwidth}{7.3in}
\setlength{\textheight}{9.3in}
\usepackage{graphicx}
\usepackage{hyperref}
\usepackage{multicol}
\usepackage{subcaption}
\usepackage[spanish]{babel}
\usepackage[utf8]{inputenc}
\usepackage{nicefrac}
\usepackage{verbatim}

\pagestyle{empty}		% Remove page number

\begin{document}

\noindent
{\Large \bf Programación Científica 1}
\vskip 0.3cm
\noindent
{\large \bf Taller 4: {\it Funciones e interface.}}
\vskip 0.2cm
\noindent
% {\it Temas: Operadores, expresiones y variables.}\\
{\it Profesores:  Gilles Pieffet}\\


% {\bf 1ra Parte.} Valores y Operadores.
\noindent
\begin{enumerate}
\item En este taller nos vamos a conectar al servidor del grupo Sistemas Complejos para entrar y correr los programas de Python. Para conectarse entre en el terminal el comando siguiente:
	\begin{verbatim}
		$ ssh -Y -o ServerAliveInterval=300 -o	ConnectTimeout=90 -l bioquimstud? 186.28.225.59
	\end{verbatim}
donde el $?$ corresponde a su posición en la lista de estudiantes:
	\begin{verbatim}
		1: ANDREA
		2: TATIANA
		3: PALOMA
		4: CARLOS A
		5: JOEL
		6: CARLOS E
		7: KAREN 
		8: LUISA 
	\end{verbatim}
y la clave para conectarse es su código.

	\item Una vez que está conectado, entre en la carpeta swampy y en adentro del interpretador python/ipython escriba el código siguiente:
		\begin{verbatim}
			from TurtleWorld import *

			world = TurtleWorld()
			bob = Turtle()
			print bob
		\end{verbatim}
Aparece un tortuga ('Bob') que se puede mover con las funciones \emph{fd} (forward/adelante), \emph{bk} (backward/atras), \emph{lt} (left turn/giro izquierda) y \emph{rt} (right turn/giro derecho). Como la tortuga tiene lápiz, puede dibujar en la ventana cuando se mueva. Entre el siguiente código para verla en acción:
		\begin{verbatim}
			fd(bob, 100)
			lt(bob)
			fd(bob, 100)
		\end{verbatim}
\item Ahora modifique el programa para dibujar un cuadrado. Cuando lo tiene dibujado, guarde en un archivo ``mypolygon.py'' todas las instrucciones que le permitieron dibujar el cuadrado (puede experimentar y dibujar otras formas también pero por lo menos tiene que tener las instrucciones para dibujar el cuadrado). Para que la ventana no se cierre automáticamente cuando se acabe el programa, puede utilizar la función siguiente al fin del programa:
		\begin{verbatim}
			wait_for_user()
		\end{verbatim}
\newpage
\item Re-escribe el programa que corresponde al dibujo del cuadrado, pero esta vez utilizando un bucle \emph{for} para remplazar instrucciones idénticas.
\vskip 0.5mm
\textbf{En los ejercicios siguientes, piense en el interés de cada una de las tareas que tiene que realizar.}

\item Escribe una función llamada \emph{cuadrado} que recibe una tortuga como argumento $t$. La función debe utilizar la tortuga para dibujar un cuadrado. Escribe una llamada de función que pasa a $bob$ como argumento de \emph{cuadrado} y corre el programa.

\item Agregue otro argumento llamado \emph{longitud} a la función \emph{cuadrado}. Modifique el cuerpo de la función de tal manera que la longitud de los lados sea \emph{longitud} y cambia la llamada de función para tener en cuenta este segundo argumento. Corre el programa otra vez y luego compruebalo con varios valores de \emph{longitud}.

\item Las funciones \emph{lt} y \emph{rt} hacen giros de $90^{\circ}$ por defecto, pero se les puede dar un segundo argumento para indicar la cantidad de grados. Por ejemplo:
		\begin{verbatim}
			lt(bob, 45)
		\end{verbatim}
gira $bob$ de $45^{\circ}$ a la izquierda.\\ 
Escribe una función \emph{poligono} basada en \emph{cuadrado} que recibe un argumento extra $n$ que corresponde al numero de lados del polígono. Modifique el cuerpo de la función de tal manera que dibuje un polígono regular con n lados.\\
Pista: los ángulos externos de un polígono regular de n~lados son de $360/n$ grados.\\
Nota: si $bob$ se mueve de manera muy lenta, lo puede acelerar cambiando \emph{bob.delay} que corresponde al tiempo entre movimientos en segundos. Un valor de $0.01$ debería ser suficiente ($bob.delay = 0$.$01$).

\item Escribe una función llamada \emph{circulo} que recibe una tortuga $t$ y un radio $r$ como argumentos y dibuja un circulo aproximado llamando a \emph{poligono} con una longitud y un numero de lados apropiados. Comprueba su función con varios valores de $r$.\\
Pista: encuentra el perímetro de un circulo y asegurase que $longitud * n = perimetro$.

\end{enumerate}
\end{document}


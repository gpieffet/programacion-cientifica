\documentclass[10pt, twocolumn]{article}
\setlength{\topmargin}{-0.7in}
\setlength{\oddsidemargin}{-0.4in} 
\setlength{\textwidth}{7.3in}
\setlength{\textheight}{9.3in}
\usepackage{graphicx}
\usepackage{hyperref}
\usepackage{multicol}
\usepackage{subcaption}
\usepackage[spanish]{babel}
\usepackage[utf8]{inputenc}
\usepackage{nicefrac}
\usepackage{verbatim}

\pagestyle{empty}		% Remove page number

\begin{document}

\noindent
{\Large \bf Programación Científica 2}
\vskip 0.3cm
\noindent
{\large \bf Taller 5: {\it Condiciones y Flujo de control}}
\vskip 0.2cm
\noindent
% {\it Temas: Operadores, expresiones y variables.}\\
{\it Profesor:  Gilles Pieffet}\\


% {\bf 1ra Parte.} Condicionales
\noindent
\begin{enumerate}
\item El Ultimo teorema de Fermat dice que no existen números enteros positivos \emph{a}, \emph{b} y \emph{c} tal que:
\[ a^n + b^n = c^n \]
para cualquier valor de n mayor a 2.
\begin{enumerate}
	\item Escriba una función \emph{chequear\_fermat} que recibe cuatro parámetros \emph{a}, \emph{b}, \emph{c} y \emph{n} y que chequee si el teorema de Fermat es valido. Si \emph{n} es mayor a $2$ y se encuentra que:
	\[ a^n + b^n = c^n \]
	el programar debe imprimir en la pantalla ``Oye, Fermat estaba equivocado!''. Si no, el programa debe imprimir ``No, la igualdad no funciona''.
	\item Escribe una función que pide al usuario los valores \emph{a}, \emph{b}, \emph{c} y \emph{n}, los convierte en \emph{int} y llama a \emph{chequear\_fermat} para ver si están en acuerdo con el teorema de Fermat.
	\item Modifique la función \emph{chequear\_fermat} de tal manera que verifique que \emph{n} es mayor a 2 (antes de chequear si el teorema de Fermat es valido).
\end{enumerate}

\item Si tiene tres palitos, no es siempre posible organizarlos en un triángulo. Por ejemplo, si uno de los palitos tiene una longitud de $12$ cm y los otros tienen un longitud de $3$ cm cada uno, es muy claro que los dos pequeños palos no se van a poder tocar. Con tres longitudes cualquiera, hay un test muy sencillo para ver si pueden formar un triángulo:\\
	\vskip -4mm
	\emph{Si una de las longitudes es mayor a la suma de las dos otras, no se puede formar un triángulo. De lo contrario, es posible formar uno (si la suma de dos longitudes es igual a la tercera, se forma lo que se llama un triángulo degenerado).}
\begin{enumerate}
	\item Escribe una función llamada \emph{es\_triangulo} que recibe tres números enteros como argumentos, y que imprime ``Si'' o ``No'' dependiendo de si se puede formar un triángulo con tres palos con estas longitudes. 
	\item Modifique el programa para tener en cuenta el caso en el cual el triángulo es degenerado. En este caso imprima ``Si, pero es degenerado'' en vez de simplemente ``Si''. Escribe la expresión condicional de tal manera que chequee si dos lados juntos son iguales al tercero.
	\item Haga lo mismo que en \emph{(b)} pero de tal manera que la expresión condicional chequee si dos lados juntos \textbf{NO} son iguales al tercero. 
	\item Escribe una función que le pide al usuario las tres longitudes y utiliza \emph{es\_triangulo} para chequear si se puede formar un triángulo con tres palitos con estas longitudes.
\end{enumerate}

\item {\textbf{Adivina el numero.}} Tiene que programar un juego en el cual el computador \emph{piensa} un numero de $1$ a $20$, y tiene $4$ intentos para adivinar el numero correcto. Abajo es lo que aparece cuando el programa está corriendo. Lo que entra el usuario está indicado con letras negrillas.
\vskip 1mm
Hola! Cual es tu nombre?\\
\textbf{Paloma}\\
Bueno Paloma, estoy pensando en un numero de $1$ a $20$.\\
Adivina cuál es.\\
\textbf{10}\\
No. Eso era el intento 1.\\
Adivina cuál es.\\
\textbf{15}\\
No. Eso era el intento 2.\\
Adivina cuál es.\\
\textbf{7}\\
Buen trabajo, Paloma. Encontraste mi número en 3 intentos.

Si no se encuentra el numero en cuatro intentos, lo que aparece después del cuarto intento es:\\
No. Eso era el intento 4.\\
El numero que tenia en la mente era el 14

Nota 1: Para que el computador escoja un numero al azar tenemos que utilizar una función que genera un numero aleatorio: la función \emph{randint} del \underline{modulo} \emph{random}. Para utilizarla, se le pasa como argumentos el primero y el ultimo numero (los limites) del rango que nos interesa, y la función devuelva un numero adentro de este rango.
\vspace{-3mm}
\begin{verbatim}
randint(1, 20)	-> numero entre 1 y 20
\end{verbatim}

Nota 2: Para salir de un bucle de manera anticipada (por ejemplo si se cumple una condición externa), se utiliza la instrucción \emph{break}.

Nota 3: Cuando está probando el programa, con el fin de evitar la perdida de tiempo asociada con la entrada manual de datos, puede utilizar valores predefinidas o que se inicializan automáticamente. Una vez que el programa funciona de manera adecuada, se puede cambiar para pedir los valores al jugador de manera interactiva.

\item {\textbf{Adivina el numero v2.}} Para ayudar el jugador a encontrar el numero correcto, indica para cada intento si su numero es mas grande o mas pequeño que el numero a encontrar:\\
Adivina cual es.\\
\textbf{18}\\
Su numero es mas grande.

\end{enumerate}
\end{document}


\documentclass[10pt, twocolumn]{article}
\setlength{\topmargin}{-0.7in}
\setlength{\oddsidemargin}{-0.4in} 
\setlength{\textwidth}{7.3in}
\setlength{\textheight}{9.3in}
\usepackage{graphicx}
\usepackage{hyperref}
\usepackage{multicol}
\usepackage{subcaption}
\usepackage[spanish]{babel}
\usepackage[utf8]{inputenc}
\usepackage{nicefrac}
\usepackage{verbatim}

\pagestyle{empty}		% Remove page number

\begin{document}

\noindent
{\Large \bf Programación Científica 2}
\vskip 0.3cm
\noindent
{\large \bf Taller 5: {\it Condiciones, Flujo de control y Recursión}}
\vskip 0.2cm
\noindent
% {\it Temas: Operadores, expresiones y variables.}\\
{\it Profesores:  Gilles Pieffet}\\


% {\bf 1ra Parte.} Condicionales
\noindent
\begin{enumerate}
\item El Ultimo teorema de Fermat dice que no existen números enteros positivos \emph{a}, \emph{b} y \emph{c} tal que:
\[ a^n + b^n = c^n \]
para cualquier valor de n mayor a 2.
\begin{enumerate}
	\item Escriba una función \emph{chequear\_fermat} que recibe cuatro parámetros \emph{a}, \emph{b}, \emph{c} y \emph{n} y que chequee si el teorema de Fermat es valido. Si \emph{n} es mayor a $2$ y se encuentra que:
	\[ a^n + b^n = c^n \]
	el programar debe imprimir en la pantalla ``Oye, Fermat estaba equivocado!''. Si no, el programa debe imprimir ``No, la igualdad no funciona''.
	\item Escribe una función que pide al usuario los valores \emph{a}, \emph{b}, \emph{c} y \emph{n}, los convierte en \emph{int} y llama a \emph{chequear\_fermat} para ver si están en acuerdo con el teorema de Fermat.
\end{enumerate}

	\item Si tiene $3$ palitos, es posible que no sea posible de organizarlos en un triángulo. Por ejemplo, si uno de los palitos tiene una longitud de $12$ cm y los otros dos tienen un longitud de $3$ cm, es muy claro que los dos pequeños palos no se van a poder tocar. Para tres longitudes cualquiera, hay un test muy sencillo para ver si pueden formar un triángulo:\\
	\vskip -4mm
	\emph{Si una de las longitudes es mayor a la suma de las dos otras, no se puede formar un triángulo. De lo contrario es posible formar uno (si la suma de $2$ longitudes es igual a la tercera, se forma lo que se llama un triángulo degenerado).}
	\begin{enumerate}
		\item Escribe una función llamada \emph{es\_triangulo} que recibe 3 números enteros como argumentos, y que imprime ``Si'' o ``No'' dependiendo de si se puede formar un triángulo con palos con estas $3$ longitudes. 
		\item Ahora, haga la distinción si el triángulo esta degenerado y en este caso imprima ``Si, pero degenerado''. Escribe la expresión condicional que va a evaluar la condición de tal manera que chequea si dos lados son igual al tercero.
		\item Haga lo mismo pero de tal manera que la expresión condicional chequea si dos lados \textbf{NO} son iguales al tercero. 
		\item Escribe una función que le pide al usuario las tres longitudes de los palos y utiliza \emph{es\_triangulo} para chequear si se puede formar un triángulo con palitos con estas $3$ longitudes.
	\end{enumerate}


% 	\begin{verbatim}
% 		$ ssh -Y -o ServerAliveInterval=300 -o	ConnectTimeout=90 -l bioquimstud? 186.28.225.59
% 	\end{verbatim}
% donde el $?$ corresponde a su posición en la lista de estudiantes:
% 	\begin{verbatim}
% 		1: ANDREA
% 		2: TATIANA
% 		3: PALOMA
% 		4: CARLOS A
% 		5: JOEL
% 		6: CARLOS E
% 		7: KAREN
% 		8: LUISA
% 	\end{verbatim}
% y la clave para conectarse es su código.

\end{enumerate}
\end{document}


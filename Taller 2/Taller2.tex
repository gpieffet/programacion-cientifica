\documentclass[10pt, twocolumn]{article}
%\documentclass[prd,amssymb,preprint]{revtex4}
\setlength{\topmargin}{-0.5in}
\setlength{\oddsidemargin}{-0.4in} 
\setlength{\textwidth}{7.3in}
\setlength{\textheight}{9.0in}
\usepackage{graphicx}
\usepackage{hyperref}
\usepackage{multicol}
\usepackage{subcaption}
\usepackage[spanish]{babel}
\usepackage[utf8]{inputenc}
\usepackage{nicefrac}
\usepackage{verbatim}

\pagestyle{empty}		% Remove page number

\begin{document}

\noindent
{\Large \bf Programación Científica 1}
\vskip 0.3cm
\noindent
{\large \bf Taller 2: {\it Operadores, expresiones y variables.}}
\vskip 0.2cm
\noindent
% {\it Temas: Operadores, expresiones y variables.}\\
{\it Profesores:  Gilles Pieffet}\\


{\bf 1ra Parte.} Valores y Operadores.
\begin{enumerate}

% Pregunta 1
% \item first item
\item Entrar lo siguiente en un archivo (utilice sublime text with \emph{stext}) y guardarlo como ``ej1\_operadores.py'' (sin las comillas) antes de correrlo (\textbf{ctrl-b}):
\begin{verbatim}
	print "Voy a contar mis pollos:"
	
	print "Gallina", 25 + 30 / 6
	print "Gallo", 100 - 25 * 3 % 4
	
	print ""
	print "Ahora voy a contar los huevos:"
	print 3 + 2 + 1 - 5 + 4 % 2 - 1 / 4 + 6
\end{verbatim}
Nota: corra el programa y mire bien lo que hace antes de seguir con el 2$^{do}$ punto.

% Pregunta 2
\item Ahora vamos a utilizar un carácter especifico al español. En el mismo archivo añade al final las lineas siguientes, y intente correr el programa de nuevo. Que pasa?
\begin{verbatim}
	print ""
	print "Ahora voy a contar piñas:"
	print "No, piñas no."
\end{verbatim}

Se puede arreglar el error colocando la linea siguiente al principio del archivo (en la  primera línea):
\begin{verbatim}
	# -*- coding: utf-8 -*-
\end{verbatim}
y luego una ``u'' al frente de todas las cadenas que contienen caracteres ofensivos:
\begin{verbatim}
	print u"Ahora voy a contar piñas:"
\end{verbatim}
Intenta de correr el programa de nuevo. Que pasa?

% Pregunta 3
\item Ahora escribe lo siguiente y corra el programa.
\begin{verbatim}
	print ""
	print "Es cierto que 3 + 2 < 5 - 7?"
	print 3 + 2 < 5 - 7

	print ""
	print "Que vale 3 + 2?", 3 + 2
	print "Que vale 5 - 7?", 5 - 7
	print "Oh, por eso es incorrecto (False)."

	print ""
	print "Unos mas."
	print "Es mas grande?", 5 > -2
	print "Es mas grande o igual?", 5 >= -2
	print "Es menos o igual?", 5 <= -2
\end{verbatim}
Que nuevos valores aparecieron? A que tipo pertenecen?

\item Prioridad de los operadores. Si un circulo tiene una circunferencia $l$ igual a 50 cm, > cual es el radio $r$?

\item División entera. The volume of a sphere of radius $r$ is $\frac43 \pi r^3$. What is the volume of a sphere of radius $5$?

\item Asumiendo que hacemos las asignaciones siguientes:
\begin{verbatim}
	width = 17
	height = 12.0
	delimiter = '.'
\end{verbatim}
Para cada una de las expresiones que sigue, escribir el valor de la expresión y el tipo (del resultado) ANTES de correr el código. Luego verifique sus respuestas con el interpretador (ipython) o desde el editor de texto (stext).
\begin{verbatim}
	1. width/2
	2. width/2.0
	3. height/3
	4. 1 + 2 * 5
	5. delimiter * 5
\end{verbatim}


\vspace{5mm}
% \newpage
{\bf 2da Parte.} Variables y strings.
\item Un programa sencillo con variables. Guarde lo siguiente en un archivo ``ej2\_variables.py'' antes de correrlo.
\begin{verbatim}
	my_name = 'escribe su nombre'
	my_age = escribe su edad (numero)
	my_height = escribe su altura (numero) # metros
	my_weight = escribe su peso (numero)	# kg
	my_eyes = 'color de los ojos'
	my_hair = 'color del cabello'
	
	print "Let's talk about", my_name
	print "He/she has a height (in meter) of", my_height
	print "He/she has a weight (in kg)", my_weight
	print "Actually that's not too heavy."
	print "His/her eyes are", my_eyes
	print "and his/her hair is", my_hair

	print "If I add my age, my height and my weight	it all add up to", my_age + my_height + my_weight
\end{verbatim}

\item Debugging. Corregir los errores en las siguientes expresiones y escribir en un comentario cual era la razón por el error en cada uno de los casos:
\begin{verbatim}
	precio carro = 20000000
	#--- Error: 
	
	no_libros = 3
	precio_libro = 15000
	precio_total = no_libro * precio_libro
	#--- Error: 
	
	latex = "LaTeX is a publishing software!"
	print "What is LaTeX?", LaTeX
	#--- Error: 
	
	altura1 = 10
	longitud2 = 5
	3area = altura1 * longitud2
	#--- Error: 

	students = 25
	class = 3
	total_students = students * class
	#--- Error: 
\end{verbatim}

\item Format string (formato de cadenas). Vamos a utilizar las mismas variables que en el programa ``ej2\_variables.py'', pero ahora vamos a incluir las variables adentro de los \emph{strings} (a través de los formatos de los tipos correspondientes) cuando utilizamos las declaraciones \emph{print}.
\begin{verbatim}
	print "Let's talk about %s." % my_name
	print "He's %f m tall." % my_height
	print "He's %d year old." % my_age
	print "He's %f kg heavy." % my_weight
	print "Actually that's not too heavy."
	print "His weight to height ratio is %.2f kg/m"	% (my_weight/my_height)
	print "He has %s eyes and %s hair." % (my_eyes	,my_hair)
	
	# this line is tricky, be careful 
	print "If I add %f, %r, and %d I get %r" % (my_	age, my_height, my_weight, my_age + my_height +	my_weight)
\end{verbatim}

\end{enumerate}
\end{document}

